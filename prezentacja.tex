\documentclass{beamer}

\usepackage[utf8]{inputenc}
\usepackage[polish]{babel}
\usepackage{amsmath, amssymb}
\usepackage{graphicx}
\usepackage{booktabs}
\usepackage{graphicx}

\title[Projekt 1, zadanie 18]{Numeryczne całkowanie po kole jednostkowym\\
z użyciem złożonej metody trapezów}
\author[Artur Czołgosz]{Artur Czołgosz, 339051\\
grupa 2a, środa 16:15-18:00}
\institute{Metody Numeryczne}
\date{Projekt 1, zadanie 18}

\begin{document}

%------------------------------------------------
% SLIDE 1: TYTUŁOWY
%------------------------------------------------
\begin{frame}
    \titlepage
\end{frame}

%------------------------------------------------
% SLIDE 2: TREŚĆ ZADANIA + CEL
%------------------------------------------------
\begin{frame}{Treść zadania i cel projektu}
    \textbf{Treść zadania (własnymi słowami):}
    \begin{itemize}
        \item Zaprojektować i zaimplementować funkcję Matlab
        \[
            I \approx \iint_{D} f(x,y)\,dx\,dy,
        \]
        gdzie $D = \{(x,y) : x^2 + y^2 \le 1\}$ (dysk jednostkowy),
        przy użyciu złożonej metody trapezów przekształcając koło jednostkowe na kwadrat: \[
            [-1,1] \times [-1,1]
        \]
    \end{itemize}

    \vspace{0.5em}
    \textbf{Cel projektu:}
    \begin{itemize}
        \item Poprawnie zaimplementować metodę numeryczną.
        \item Przetestować poprawność programu na funkcjach z dokładnie znaną całką.
        \item Zbadać własności numeryczne metody (zbieżność, wpływ mapowania).
    \end{itemize}
\end{frame}

%------------------------------------------------
% SLIDE 3: OPIS OBSZARU I ZMIANY ZMIENNYCH
%------------------------------------------------
\begin{frame}{Opis matematyczny: obszar całkowania}
    \textbf{Obszar:} dysk jednostkowy
    \[
        D = \{(x,y)\in\mathbb{R}^2 : x^2 + y^2 \le 1\}.
    \]

    Przechodzimy do współrzędnych biegunowych:
    \[
        x = r\cos\theta,\qquad y = r\sin\theta,\qquad
        r \in [0,1],\ \theta \in [0,2\pi],
    \]
    \[
        dx\,dy = r\,dr\,d\theta.
    \]

    \textbf{Równoważna całka:}
    \[
        \iint_{D} f(x,y)\,dx\,dy
        = \int_{0}^{2\pi} \int_{0}^{1} f(r\cos\theta, r\sin\theta)\, r\,dr\,d\theta.
    \]
\end{frame}

%------------------------------------------------
% SLIDE 4: MAPOWANIE KWADRATU NA (r,θ)
%------------------------------------------------
\begin{frame}{Opis matematyczny: mapowanie kwadratu}
    Wprowadzamy siatkę w zmiennych pomocniczych $(u,v)$:
    \[
        u \in [-1,1],\quad v \in [-1,1].
    \]

    Przykładowe mapowanie:
    \begin{align*}
        r(u) &= \frac{u+1}{2} \quad \text{(mapowanie \texttt{linear})},\\[0.3em]
        r(u) &= \sqrt{\frac{u+1}{2}} \quad \text{(mapowanie \texttt{sqrt})},\\[0.3em]
        \theta(v) &= \pi(v+1).
    \end{align*}

    \textbf{Jacobian całkowity:}
    \[
        J(u,v) = \left|\det \frac{\partial(x,y)}{\partial(u,v)}\right|
        = r(u)\,\left|\frac{dr}{du}\right|\,\left|\frac{d\theta}{dv}\right|.
    \]

    Zastąpienie $(x,y)$ przez $(u,v)$ prowadzi do całki:
    \[
        \iint_{D} f(x,y)\,dx\,dy
        = \int_{-1}^{1}\!\!\int_{-1}^{1}
            f\bigl(x(u,v),y(u,v)\bigr)\,J(u,v)\,du\,dv.
    \]
\end{frame}

\begin{frame}[t]{Mapowanie \texttt{linear}}
\small
\[
  (u,v) \in [-1,1]^2,\qquad \theta(v) = \pi v.
\]

\textbf{Definicja:}
\[
  r(u) = \frac{u+1}{2},\qquad
  x = r(u)\cos\theta(v),\quad
  y = r(u)\sin\theta(v).
\]

\textbf{Jacobian:}
\[
  J(u,v) = r(u)\,\frac{dr}{du}\,\frac{d\theta}{dv},
  \quad \frac{dr}{du} = \frac{1}{2},\ 
  \frac{d\theta}{dv} = \pi,
\]
\[
  J_{\text{linear}}(u,v)
  = r(u)\cdot\frac{1}{2}\cdot\pi
  = \frac{\pi}{2}r(u)
  = \frac{\pi}{4}(u+1).
\]
\end{frame}


\begin{frame}[t]{Mapowanie \texttt{sqrt}}
\small
\[
  (u,v) \in [-1,1]^2,\qquad \theta(v) = \pi v.
\]

\textbf{Definicja:}
\[
  r(u) = \sqrt{\frac{u+1}{2}},\qquad
  x = r(u)\cos\theta(v),\quad
  y = r(u)\sin\theta(v).
\]

\textbf{Jacobian:}
\[
  J(u,v) = r(u)\,\frac{dr}{du}\,\frac{d\theta}{dv},
  \quad \frac{dr}{du} = \frac{1}{4r(u)},\ 
  \frac{d\theta}{dv} = \pi,
\]
\[
  J_{\text{sqrt}}(u,v)
  = r(u)\cdot\frac{1}{4r(u)}\cdot\pi
  = \frac{\pi}{4}\quad\text{(stała)}.
\]

\begin{itemize}\itemsep2pt
  \item Stały Jacobian $\Rightarrow$ równomierne „zagęszczenie” w promieniu.
\end{itemize}
\end{frame}


\begin{frame}[t]{Mapowanie \texttt{m1} (niepoprawne)}
\small
\textbf{Definicja:}
\[
  x(u,v) = u,\qquad
  y(u,v) = v\sqrt{1-u^2}.
\]

\textbf{Jacobian:}
\[
  \frac{\partial x}{\partial u} = 1,\ 
  \frac{\partial x}{\partial v} = 0,\ 
  \frac{\partial y}{\partial v} = \sqrt{1-u^2},\ 
  \frac{\partial y}{\partial u}
   = v\frac{-u}{\sqrt{1-u^2}},
\]
\[
  J_{\text{m1}}(u,v)
  = \left|\det \frac{\partial(x,y)}{\partial(u,v)}\right|
  = \sqrt{1-u^2}.
\]

\begin{block}{\alert{Uwaga} (krótko)}
\footnotesize
Mapowanie \texttt{m1} nie spełnia założeń „bezpiecznego”
przekształcenia z wykładu (nie zachowuje gładkości na całym obszarze),
dlatego traktujemy je tylko jako przykład eksperymentalny.
\end{block}
\end{frame}


\begin{frame}[t]{Mapowanie \texttt{m3}}
\small
\textbf{Definicja:}
\[
  x(u,v) = u\sqrt{1-\frac{v^2}{2}},\qquad
  y(u,v) = v\sqrt{1-\frac{u^2}{2}}.
\]
Oznaczamy
\[
  A(v) = \sqrt{1-\frac{v^2}{2}},\quad
  B(u) = \sqrt{1-\frac{u^2}{2}},
\]
więc $x = uA,\ y = vB$.

\textbf{Jacobian:}
\[
  \frac{\partial x}{\partial u} = A,\ 
  \frac{\partial x}{\partial v} = -\frac{uv}{2A},\ 
  \frac{\partial y}{\partial v} = B,\ 
  \frac{\partial y}{\partial u} = -\frac{uv}{2B},
\]
\[
  J_{\text{m3}}(u,v)
  = A B - \frac{u^2 v^2}{4AB}.
\]

\[
  J_{\text{m3}}(u,v)
  = \frac{1 - \dfrac{u^2+v^2}{2}}
         {\sqrt{1-\dfrac{u^2}{2}}\,
          \sqrt{1-\dfrac{v^2}{2}}}.
\]
\end{frame}

%------------------------------------------------
% SLIDE 5: ZŁOŻONA METODA TRAPEZÓW
%------------------------------------------------
\begin{frame}{Opis matematyczny: złożona metoda trapezów}
    Na prostokącie $[-1,1]\times[-1,1]$ stosujemy siatkę:
    \[
        u_i = -1 + i\,h_u,\quad i=0,\dots, N_u,
        \qquad h_u = \frac{2}{N_u},
    \]
    \[
        v_j = -1 + j\,h_v,\quad j=0,\dots, N_v,
        \qquad h_v = \frac{2}{N_v}.
    \]

    \textbf{Wzór złożonej metody trapezów 2D:}
    \[
        I_{N_u,N_v} =
        \sum_{i=0}^{N_u} \sum_{j=0}^{N_v}
          w_i^{(u)}\, w_j^{(v)}\,
          f\bigl(x_{ij},y_{ij}\bigr)\, J_{ij}\, h_u h_v,
    \]
    gdzie
    \[
        (x_{ij},y_{ij}) = (x(u_i,v_j),y(u_i,v_j)),
    \]
    a $w_i^{(u)}, w_j^{(v)} \in \{1,\tfrac{1}{2}\}$ są wagami 1D metody trapezów.
\end{frame}

%------------------------------------------------
% SLIDE 6: WŁASNOŚCI METODY
%------------------------------------------------
\begin{frame}{Własności metody}
    \begin{itemize}
        \item Metoda trapezów 1D ma rząd zbieżności $O(h^2)$
              przy dostatecznej gładkości funkcji.
        \item Dla 2D na prostokącie: błąd rzędu
              \[
                  |I - I_{N_u,N_v}| = O(h_u^2 + h_v^2).
              \]
        \item Jakość przybliżenia na kole zależy od:
        \begin{itemize}
            \item gładkości funkcji $f$ na dysku,
            \item gładkości użytego mapowania $(u,v)\mapsto (x,y)$,
            \item zagęszczenia siatki ($N_u$, $N_v$).
        \end{itemize}
        \item Porównujemy różne mapowania (\texttt{linear}, \texttt{sqrt}, ...),
              obserwując wpływ na błąd i zbieżność.
    \end{itemize}
\end{frame}

%------------------------------------------------
% SLIDE 7: TESTY POPRAWNOŚCI – ZAŁOŻENIA
%------------------------------------------------
\begin{frame}{Testy poprawności programu}
    \textbf{Idea:} testujemy program na funkcjach,
    dla których znamy dokładną wartość całki po dysku.

    \textbf{Przykładowe funkcje testowe:}
    \begin{itemize}
        \item $f_1(x,y) = 1$,\quad
              $\displaystyle I = \iint_{D} 1\,dx\,dy = \pi$.
        \item $f_2(x,y) = x$, $f_3(x,y) = y$,\quad
              całka równa 0 (symetria).
        \item $f_4(x,y) = x^2 + y^2 = r^2$,\quad
              $\displaystyle I = \int_0^{2\pi}\!\!\int_0^1 r^2 \cdot r\,dr\,d\theta 
              = \frac{\pi}{2}$.
        \item (opcjonalnie) inne wielomiany lub funkcje gładkie.
    \end{itemize}

    Na slajdach z testami pokazujemy:
    \begin{itemize}
        \item nazwę funkcji $f$,
        \item dokładną wartość całki,
        \item wartości numeryczne dla różnych $N$,
        \item błąd bezwzględny.
    \end{itemize}
\end{frame}

%------------------------------------------------
% SLIDE 8: TESTY POPRAWNOŚCI – PRZYKŁADOWA TABELA
%------------------------------------------------
\begin{frame}{Testy poprawności: przykład dla $f(x,y)=1$}
    \begin{table}
        \centering
        \begin{tabular}{cccc}
            \toprule
            $N_u=N_v$ & $I_{\text{num}}$ & $I_{\text{dokładne}}$ & $|błąd|$ \\
            \midrule
            4   & 3.1416 & 3.1416 & $<10^{-4}$ \\
            8   & 3.1416 & 3.1416 & $<10^{-6}$ \\
            16  & 3.1416 & 3.1416 & $<10^{-8}$ \\
            \bottomrule
        \end{tabular}
    \end{table}

    \vspace{0.5em}
    \textbf{Wnioski:}
    \begin{itemize}
        \item Program daje poprawne wyniki dla prostych funkcji.
        \item Błąd maleje przy zagęszczaniu siatki.
        \item Na tym etapie przede wszystkim potwierdzamy
              poprawność \emph{implementacji}.
    \end{itemize}
\end{frame}

\begin{frame}[t]{Testy poprawności: $f(x,y)=1$ (linear, sqrt)}
\scriptsize
Dokładna wartość: $I_{\text{exact}} = \pi$.

\medskip
\textbf{Mapping \texttt{linear}}
\renewcommand{\arraystretch}{0.95}
\begin{tabular}{ccc}
\toprule
$N_u=N_v$ & $I_{\text{num}}$ & $|I_{\text{num}}-I_{\text{exact}}|$ \\
\midrule
4  & 3.14159265358979 & 0 \\
8  & 3.14159265358979 & $4.44\cdot 10^{-16}$ \\
16 & 3.14159265358979 & 0 \\
32 & 3.14159265358979 & 0 \\
64 & 3.14159265358979 & 0 \\
\bottomrule
\end{tabular}

\medskip
\textbf{Mapping \texttt{sqrt}}
\renewcommand{\arraystretch}{0.95}
\begin{tabular}{ccc}
\toprule
$N_u=N_v$ & $I_{\text{num}}$ & $|I_{\text{num}}-I_{\text{exact}}|$ \\
\midrule
4  & 3.14159265358979 & 0 \\
8  & 3.14159265358979 & $1.78\cdot 10^{-15}$ \\
16 & 3.14159265358980 & $3.11\cdot 10^{-15}$ \\
32 & 3.14159265358979 & $1.78\cdot 10^{-15}$ \\
64 & 3.14159265358980 & $3.11\cdot 10^{-15}$ \\
\bottomrule
\end{tabular}

\medskip
\footnotesize
W obu przypadkach błąd jest na poziomie błędu zaokrągleń.
\end{frame}

\begin{frame}[t]{Testy poprawności: funkcje z zerową całką}
\scriptsize
Dokładna wartość: $I_{\text{exact}} = 0$.

\medskip
\textbf{$f(x,y)=x$, mapping \texttt{linear}}
\renewcommand{\arraystretch}{0.95}
\begin{tabular}{ccc}
\toprule
$N_u=N_v$ & $I_{\text{num}}$ & $|I_{\text{num}}|$ \\
\midrule
4  & $3.21\cdot 10^{-17}$ & $3.21\cdot 10^{-17}$ \\
8  & $5.97\cdot 10^{-17}$ & $5.97\cdot 10^{-17}$ \\
16 & $8.42\cdot 10^{-17}$ & $8.42\cdot 10^{-17}$ \\
32 & $8.46\cdot 10^{-17}$ & $8.46\cdot 10^{-17}$ \\
64 & $9.84\cdot 10^{-17}$ & $9.84\cdot 10^{-17}$ \\
\bottomrule
\end{tabular}

\medskip
\textbf{$f(x,y)=y$, mapping \texttt{sqrt}}
\renewcommand{\arraystretch}{0.95}
\begin{tabular}{ccc}
\toprule
$N_u=N_v$ & $I_{\text{num}}$ & $|I_{\text{num}}|$ \\
\midrule
4  & $-1.52\cdot 10^{-18}$ & $1.52\cdot 10^{-18}$ \\
8  & $9.71\cdot 10^{-18}$  & $9.71\cdot 10^{-18}$ \\
16 & $2.08\cdot 10^{-17}$  & $2.08\cdot 10^{-17}$ \\
32 & $7.46\cdot 10^{-18}$  & $7.46\cdot 10^{-18}$ \\
64 & $1.48\cdot 10^{-18}$  & $1.48\cdot 10^{-18}$ \\
\bottomrule
\end{tabular}

\medskip
\footnotesize
Błąd jest na poziomie precyzji maszynowej — test symetrii zdany.
\end{frame}


\begin{frame}[t]{Test poprawności: $f(x,y)=|x|$ (linear)}
\scriptsize
Dokładna wartość: $I_{\text{exact}} = \dfrac{4}{3} \approx 1.333333\ldots$

\medskip
\renewcommand{\arraystretch}{0.95}
\begin{tabular}{ccc}
\toprule
$N_u=N_v$ & $I_{\text{num}}$ & $|I_{\text{num}}-I_{\text{exact}}|$ \\
\midrule
16  & 1.3296958445 & $3.64\cdot 10^{-3}$ \\
32  & 1.3324248703 & $9.08\cdot 10^{-4}$ \\
64  & 1.3331062744 & $2.27\cdot 10^{-4}$ \\
128 & 1.3332765721 & $5.68\cdot 10^{-5}$ \\
256 & 1.3333191433 & $1.42\cdot 10^{-5}$ \\
512 & 1.3333297858 & $3.55\cdot 10^{-6}$ \\
\bottomrule
\end{tabular}

\medskip
\footnotesize
Funkcja nie jest gładka na $x=0$ $\Rightarrow$ zbieżność wolniejsza niż dla funkcji gładkich.
\end{frame}


\begin{frame}[t]{Test: $f(x,y)=x^2+y^2$ (linear)}
\scriptsize
Dokładna wartość: $I_{\text{exact}} = 1.5707963268$.

\medskip
\renewcommand{\arraystretch}{0.95}
\begin{tabular}{ccc}
\toprule
$N_u=N_v$ & $I_{\text{num}}$ & $|I_{\text{num}}-I_{\text{exact}}|$ \\
\midrule
4  & 1.5953400194 & $2.45\cdot 10^{-2}$ \\
8  & 1.5769322499 & $6.14\cdot 10^{-3}$ \\
16 & 1.5723303076 & $1.53\cdot 10^{-3}$ \\
32 & 1.5711798220 & $3.83\cdot 10^{-4}$ \\
64 & 1.5708922006 & $9.59\cdot 10^{-5}$ \\
\bottomrule
\end{tabular}

\medskip
\footnotesize
Błąd maleje ok. 4 razy przy podwojeniu $N$ (zbieżność $\sim h^2$).
\end{frame}

\begin{frame}[t]{Test: $f(x,y)=x^2+y^2$ (sqrt, m3)}
\scriptsize
Dokładna wartość: $I_{\text{exact}} = 1.5707963268$.

\medskip
\textbf{Mapping \texttt{sqrt}}
\renewcommand{\arraystretch}{0.95}
\begin{tabular}{ccc}
\toprule
$N_u=N_v$ & $I_{\text{num}}$ & $|I_{\text{num}}-I_{\text{exact}}|$ \\
\midrule
4  & 1.5707963268 & 0 \\
8  & 1.5707963268 & $2.22\cdot 10^{-16}$ \\
16 & 1.5707963268 & 0 \\
32 & 1.5707963268 & 0 \\
64 & 1.5707963268 & 0 \\
\bottomrule
\end{tabular}

\medskip
\textbf{Mapping \texttt{m3}}
\renewcommand{\arraystretch}{0.95}
\begin{tabular}{ccc}
\toprule
$N_u=N_v$ & $I_{\text{num}}$ & $|I_{\text{num}}-I_{\text{exact}}|$ \\
\midrule
4  & 1.5594268746 & $1.14\cdot 10^{-2}$ \\
8  & 1.5678594434 & $2.94\cdot 10^{-3}$ \\
16 & 1.5700554165 & $7.41\cdot 10^{-4}$ \\
32 & 1.5706106664 & $1.86\cdot 10^{-4}$ \\
64 & 1.5707498844 & $4.64\cdot 10^{-5}$ \\
\bottomrule
\end{tabular}

\medskip
\footnotesize
Dla funkcji radialnej \texttt{sqrt} jest praktycznie idealne.
\end{frame}


\begin{frame}[t]{Test: $f(x,y)=x^2$ (linear)}
\scriptsize
Dokładna wartość: $I_{\text{exact}} = 0.7853981634$.

\medskip
\renewcommand{\arraystretch}{0.95}
\begin{tabular}{ccc}
\toprule
$N_u=N_v$ & $I_{\text{num}}$ & $|I_{\text{num}}-I_{\text{exact}}|$ \\
\midrule
4  & 0.7976700097 & $1.23\cdot 10^{-2}$ \\
8  & 0.7884661250 & $3.07\cdot 10^{-3}$ \\
16 & 0.7861651538 & $7.67\cdot 10^{-4}$ \\
32 & 0.7855899110 & $1.92\cdot 10^{-4}$ \\
64 & 0.7854461003 & $4.79\cdot 10^{-5}$ \\
\bottomrule
\end{tabular}

\medskip
\footnotesize
Znów widać zbieżność $\sim h^2$ dla mapowania \texttt{linear}.
\end{frame}

\begin{frame}[t]{Test: $f(x,y)=x^2$ (sqrt, m3)}
\scriptsize
Dokładna wartość: $I_{\text{exact}} = 0.7853981634$.

\medskip
\textbf{Mapping \texttt{sqrt}}
\renewcommand{\arraystretch}{0.95}
\begin{tabular}{ccc}
\toprule
$N_u=N_v$ & $I_{\text{num}}$ & $|I_{\text{num}}-I_{\text{exact}}|$ \\
\midrule
4  & 0.7853981634 & 0 \\
8  & 0.7853981634 & 0 \\
16 & 0.7853981634 & 0 \\
32 & 0.7853981634 & 0 \\
64 & 0.7853981634 & 0 \\
\bottomrule
\end{tabular}

\medskip
\textbf{Mapping \texttt{m3}}
\renewcommand{\arraystretch}{0.95}
\begin{tabular}{ccc}
\toprule
$N_u=N_v$ & $I_{\text{num}}$ & $|I_{\text{num}}-I_{\text{exact}}|$ \\
\midrule
4  & 0.7797134373 & $5.68\cdot 10^{-3}$ \\
8  & 0.7839297217 & $1.47\cdot 10^{-3}$ \\
16 & 0.7850277082 & $3.70\cdot 10^{-4}$ \\
32 & 0.7853053332 & $9.28\cdot 10^{-5}$ \\
64 & 0.7853749422 & $2.32\cdot 10^{-5}$ \\
\bottomrule
\end{tabular}

\medskip
\footnotesize
\texttt{sqrt} jest tu znowu najdokładniejsze, \texttt{m3} stopniowo dochodzi do wyniku.
\end{frame}


%------------------------------------------------
% SLIDE 9: EKSPERYMENTY NUMERYCZNE – PLAN
%------------------------------------------------
\begin{frame}{Eksperymenty numeryczne – plan}
    \textbf{Cel eksperymentów:} zbadać własności numeryczne metody:
    \begin{itemize}
        \item zbieżność błędu w zależności od $N_u$, $N_v$,
        \item porównanie różnych mapowań (\texttt{linear}, \texttt{sqrt}, ...),
        \item wpływ kształtu i gładkości funkcji $f$.
    \end{itemize}

    \textbf{Przykładowe eksperymenty:}
    \begin{itemize}
        \item Analiza zbieżności dla kilku funkcji gładkich
              (np. $f(x,y)=1,\ x^2+y^2,\ \sin x \cos y$).
        \item Porównanie błędu dla różnych mapowań przy tym samym $N$.
        \item Zestawienie czasu obliczeń w zależności od wielkości siatki.
    \end{itemize}
\end{frame}

%------------------------------------------------
% SLIDE 10: EKSPERYMENT – ZBIEŻNOŚĆ BŁĘDU
%------------------------------------------------
\begin{frame}{Eksperyment: zbieżność błędu}
    \textbf{Przykład:} funkcja $f(x,y) = x^2 + y^2$,
    mapowanie \texttt{linear}.

    \begin{table}
        \centering
        \begin{tabular}{cccc}
            \toprule
            $N$ & $I_{\text{num}}$ & $|błąd|$ & $|błąd_{poprz}|/|błąd|$ \\
            \midrule
            4   & \dots & \dots & -- \\
            8   & \dots & \dots & $\approx 4$ \\
            16  & \dots & \dots & $\approx 4$ \\
            32  & \dots & \dots & $\approx 4$ \\
            \bottomrule
        \end{tabular}
    \end{table}

    \vspace{0.5em}
    \begin{itemize}
        \item Stosunek błędów $\approx 4$ będzie wskazywać na rząd zbieżności
              zbliżony do $O(h^2)$ (podwojenie liczby podziałów
              zmniejsza błąd około czterokrotnie).
    \end{itemize}
\end{frame}

%------------------------------------------------
% SLIDE 11: EKSPERYMENT – PORÓWNANIE MAPOWAŃ
%------------------------------------------------
\begin{frame}{Eksperyment: porównanie mapowań}
    \textbf{Przykład:} funkcja gładka $f(x,y)=x^2 + y^2$,
    porównanie mapowań przy tym samym $N$.

    \begin{table}
        \centering
        \begin{tabular}{ccc}
            \toprule
            Mapowanie & $I_{\text{num}}$ & $|błąd|$ \\
            \midrule
            \texttt{linear} & \dots & \dots \\
            \texttt{sqrt}   & \dots & \dots \\
            \texttt{m1}     & \dots & \dots \\
            \texttt{m3}     & \dots & \dots \\
            \bottomrule
        \end{tabular}
    \end{table}
\end{frame}

%------------------------------------------------
% SLIDE 12: EKSPERYMENT – ZBIEŻNOŚĆ BŁĘDU (TABELA)
%------------------------------------------------
\begin{frame}{Funkcje testowe i wartości całek}
    Rozważane funkcje testowe i ich wartości na dysku jednostkowym:
    \[
        I_k = \iint_D f_k(x,y)\,dx\,dy
    \]
    \begin{table}
        \centering
        \begin{tabular}{ccc}
            \toprule
            Funkcja & $f(x,y)$ & $I = \displaystyle\iint_D f(x,y)\,dx\,dy$ \\
            \midrule
            $f_1$ & $1$ & $\pi$ \\ [0.8em]
            $f_2$ & $x$ & $0$ \\ [0.8em]
            $f_3$ & $|x|$ & $\dfrac{4}{3}$ \\ [0.8em]
            $f_4$ & $x^2 + y^2$ & $\dfrac{\pi}{2}$ \\ [0.8em]
            $f_5$ & $x^2$ & $\dfrac{\pi}{4}$ \\
            \bottomrule
        \end{tabular}
    \end{table}
\end{frame}

%------------------------------------------------
% SLIDE 13: Funkcja f(x,y) = 1
%------------------------------------------------
\begin{frame}{Porównanie mapowań dla $f(x,y)=1$}
    \centering
    \begin{minipage}{0.48\textwidth}
        \centering
        \includegraphics[width=\textwidth]{f1_linear}
        
        \small mapowanie \texttt{linear}
    \end{minipage}
    \hfill
    \begin{minipage}{0.48\textwidth}
        \centering
        \includegraphics[width=\textwidth]{f1_m3}
        
        \small mapowanie \texttt{m3}
    \end{minipage}
\end{frame}

%------------------------------------------------
% SLIDE 14: Funkcja f(x,y) = x
%------------------------------------------------
\begin{frame}{Porównanie mapowań dla $f(x,y)=x$}
    \centering
    \begin{minipage}{0.48\textwidth}
        \centering
        \includegraphics[width=\textwidth]{f2_linear}
        
        \small mapowanie \texttt{linear}
    \end{minipage}
    \hfill
    \begin{minipage}{0.48\textwidth}
        \centering
        \includegraphics[width=\textwidth]{f2_root}
        
        \small mapowanie \texttt{sqrt}
    \end{minipage}
\end{frame}

%------------------------------------------------
% SLIDE 15: Funkcja f(x,y) = |x|
%------------------------------------------------
\begin{frame}{Porównanie mapowań dla $f(x,y)=|x|$}
    \centering
    \begin{minipage}{0.48\textwidth}
        \centering
        \includegraphics[width=\textwidth]{f3_linear}
        
        \small mapowanie \texttt{linear}
    \end{minipage}
    \hfill
    \begin{minipage}{0.48\textwidth}
        \centering
        \includegraphics[width=\textwidth]{f3_m3}
        
        \small mapowanie \texttt{m3}
    \end{minipage}
\end{frame}

%------------------------------------------------
% SLIDE 16: Funkcja f(x,y) = x^2 + y^2
%------------------------------------------------
\begin{frame}{Porównanie mapowań dla $f(x,y)=x^2 + y^2$}
    \centering
    \begin{minipage}{0.48\textwidth}
        \centering
        \includegraphics[width=\textwidth]{f4_linear}
        
        \small mapowanie \texttt{linear}
    \end{minipage}
    \hfill
    \begin{minipage}{0.48\textwidth}
        \centering
        \includegraphics[width=\textwidth]{f4_m3}
        
        \small mapowanie \texttt{m3}
    \end{minipage}
\end{frame}

%------------------------------------------------
% SLIDE 17: Funkcja f(x,y) = x^2
%------------------------------------------------
\begin{frame}{Porównanie mapowań dla $f(x,y)=x^2$}
    \centering
    \begin{minipage}{0.48\textwidth}
        \centering
        \includegraphics[width=\textwidth]{f5_linear}
        
        \small mapowanie \texttt{linear}
    \end{minipage}
    \hfill
    \begin{minipage}{0.48\textwidth}
        \centering
        \includegraphics[width=\textwidth]{f5_m3}
        
        \small mapowanie \texttt{m3}
    \end{minipage}
\end{frame}

%------------------------------------------------
% SLIDE 18: Komentarze do wykresów
%------------------------------------------------
\begin{frame}{Komentarze do wykresów}
    \begin{itemize}
        \item Dla wszystkich funkcji za pomocą metody \texttt{linear} obserwujemy zbieżność
              rzędu około $O(h^2)$, zgodnie z teorią.
        \item Na wykresach nie pokazano wszystkich mapowań.
        \item Nie pokazano dla wszystkich funkcji metody \texttt{sqrt},
                ale dla niektórych funkcji od samego początku dawała dokładny wynik.
    \end{itemize}
\end{frame}

%------------------------------------------------
% SLIDE 18: PODSUMOWANIE
%------------------------------------------------
\begin{frame}{Podsumowanie}
    \begin{itemize}
        \item Zaimplementowano złożoną metodę trapezów
              do całkowania po kole jednostkowym z użyciem
              różnych mapowań kwadratu na dysk.
        \item Testy poprawności na prostych funkcjach
              potwierdziły poprawność implementacji.
        \item Eksperymenty numeryczne pokazały oczekiwany
              rząd zbieżności $O(h^2)$.
        \item Porównanie mapowań wskazało, że wybór
              mapowania ma istotny wpływ na dokładność
              dla niektórych funkcji.
    \end{itemize}
\end{frame}

\end{document}